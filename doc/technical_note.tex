\documentclass{article}

\usepackage{amsmath}

\newcommand{\dd}{\partial}

\title{Technical notes}
\author{Iurii Karpenko}

\begin{document}
\maketitle

\section{Ideal-viscous splitting (in Cartesian coordinates)}

Here I explain how an updated method of ideal-viscous splitting is implemented in \texttt{vHLLE}. To derive it, one starts from the hydro equations:

 \begin{align}\label{eqns1}
 \dd_\nu T^{\mu\nu}=0
 \end{align}
The numerical finite volume representation of the emergy-momentum conservation part is:
\begin{align}\label{eq-num1}
&\frac{1}{\Delta t}\left[ (Q^\mu_{id,n+1}+\delta Q^\mu_{n+1}) - (Q^\mu_{id,n}+\delta Q^\mu_n) \right] \nonumber\\
&+\sum_{\alpha=1...3}\frac{1}{\Delta x_\alpha}\left[ F^{\mu\alpha}_{id,i+1/2}+\delta F^{\mu\alpha}_{i+1/2} - F^{\mu\alpha}_{id,i-1/2}-\delta F^{\mu\alpha}_{i-1/2} \right] \nonumber\\
&=(S^\mu_{n+1/2}+\delta S^\mu_{n+1/2}).
\end{align}
where second order accurate method is assumed (therefore half-step and cell-edge values), and $\delta Q$ and $\delta F$ denote viscous corrections to conserved variables and fluxes, respectively.\\
The terms in Eqs.~\ref{eq-num1} can be rearranged as follows:
\begin{align}
&\left[ Q^\mu_{id,n+1} - Q^\mu_{id,n} \right]
+\sum_{\alpha}\frac{\Delta t}{\Delta x_\alpha}\left[ F^{\mu\alpha}_{id,i+1/2}+\delta F^{\mu\alpha}_{i+1/2} - F^{\mu\alpha}_{id,i-1/2}-\delta F^{\mu\alpha}_{i-1/2} \right] \nonumber\\
&=\Delta t(S^\mu_{n+1/2}+\delta S^\mu_{n+1/2}) + (\delta Q^\mu_n - \delta Q^\mu_{n+1})
\end{align}
The above form of equations mean that that in basic hydrodynamic (i.e. energy-momentum conservation) equations one can follow the evolution of ideal part of the conserved variables $Q_{id}^\mu$ only, when extra source terms $(\delta Q^\mu_n - \delta Q^\mu_{n+1})/\Delta t$ are included in their numerical evolution equations.

\begin{thebibliography}{0}
 \bibitem{Karpenko:2013wva}
  I.~Karpenko, P.~Huovinen and M.~Bleicher,
  %``A 3+1 dimensional viscous hydrodynamic code for relativistic heavy ion collisions,''
  Comput.\ Phys.\ Commun.\  {\bf 185} (2014) 3016
  %[arXiv:1312.4160 [nucl-th]].
\end{thebibliography}

\end{document}
